\documentclass[a4paper,11pt]{article}
\usepackage{geometry}
\geometry{a4paper, margin=1in}
\usepackage{fancyhdr}
\usepackage{graphicx}
\usepackage{amsmath}
\usepackage{float}
\usepackage{hyperref}

% Header and Footer
\pagestyle{fancy}
\fancyhead[L]{Department of Engineering Coursework}
\fancyhead[R]{Page \thepage\ of 8}
\fancyfoot[C]{}
\fancyfoot[R]{ENGI 2191 Coursework Report}

\begin{document}

% Title Page
\begin{center}
    \LARGE \textbf{Electrical Engineering 2: Coursework Report} \\
    \vspace{0.2cm}
    \large Module Code: ENGI 2191 \\
    \vspace{0.2cm}
    GROUP B \\
    \vspace{0.2cm}
    Term: Michaelmas \\
    Academic Year: 2024-2025 \\
\end{center}
\vspace{1cm}

\noindent\textbf{Time Required:} \\
It is expected that you should spend approximately 40 hours on this coursework assignment, including lectures, workshops, and completing Problem Sheets.

\vspace{0.5cm}
\noindent\textbf{Deadline(s) for Submission:} Monday, 13 January 2025 at 14:00 hrs \\
\noindent\textbf{Date for Feedback:} Monday, 10 February 2025

\vspace{0.5cm}
\noindent\textbf{Submission Instructions:} \\
\begin{itemize}
    \item Submit via Learn Ultra/TURNITIN before the deadline.
    \item All submissions are electronic; no hard copy is required.
    \item Max file size: 20 MB.
    \item Save using the naming convention: SURNAME-Firstname\_ENGI2191.pdf (e.g., BLOGGS-Joanne\_ENGI2191.pdf).
\end{itemize}

\vspace{0.5cm}
\noindent\textbf{Format:} \\
\begin{itemize}
    \item Submit in PDF format, max 8 pages (including diagrams, references).
    \item Appendices may be included but won’t be examined or count toward the page limit.
\end{itemize}

\vspace{0.5cm}
\noindent\textbf{Penalties for Non-Compliance:} \\
Submissions must meet format specifications. Incorrect file formats won’t be marked.

\vspace{0.5cm}
\noindent\textbf{Late Submission:} \\
In line with the Learning and Teaching Handbook, submissions up to five days late are capped at the module pass mark; beyond five days, a mark of zero is recorded.

\vspace{0.5cm}
\noindent\textbf{Academic Integrity Guidance:} \\
The Department of Engineering investigates potential academic dishonesty per the Learning and Teaching Handbook, Section 6.2.4: Academic Misconduct.

\vspace{0.5cm}
\noindent\textbf{Use of Generative AI:} \\
Generative AI use is allowed only where explicitly stated. Any use of gAI must be acknowledged in your submission.

\vspace{1cm}
\section*{AHEP Learning Outcomes Assessed}
This assignment assesses the following AHEP Learning Outcomes:
\begin{itemize}
    \item \textbf{M12.} Use practical laboratory and workshop skills to investigate complex problems.
\end{itemize}

\vspace{1cm}
\section*{Instructions to Candidates}
Include the assignment title and your full name on the first page. Name your submission as: SURNAME-firstname\_ENGI2191.pdf.

\vspace{1cm}
\section*{Coursework Brief: Introduction}
This coursework is part of module ENGI 2191 (Electrical Engineering II) and contributes 20\% toward the module marks. You will analyze a three-phase system with three different loads, calculating current, power, voltage drops, and losses, and assessing the effect of a capacitive bank at the transmission line’s end.

\vspace{0.5cm}
Refer to the lecture notes and recommended texts in the Bill Bryson Library for theory.

\vspace{1cm}
\section*{Coursework Exercise}
This assignment has four sections. Answer all sections, including theory, calculations, results, and discussions. The suggested length is 6 to 8 pages, excluding cover and references, using an 11-point font.

\vspace{1cm}
\noindent\textbf{1. Current and Voltage Analysis:} \\
Calculate the phasor of currents and voltages in the network.

\vspace{0.5cm}
\noindent\textbf{2. Power and Power Factor Analysis:} \\
Use section 1 results to analyze network power, including active, reactive, and power factors.

\vspace{0.5cm}
\noindent\textbf{3. Power Losses and Voltage Drop Analysis:} \\
Calculate voltage drops and power losses across each transmission line.

\vspace{0.5cm}
\noindent\textbf{4. Reactive Power Compensation:} \\
Install a capacitive bank at terminal T3 and assess the effect on system power factor.

\vspace{1cm}
\section*{Marking and Feedback Matrix}
\begin{itemize}
    \item \textbf{Presentation of the Report (25\%)}: Quality and clarity of visual presentation and references.
    \item \textbf{Figures (15\%)}: Use of graphics to support the report.
    \item \textbf{Results and Discussions (30\%)}: Quality of solutions and engineering interpretation.
    \item \textbf{Technical Content (30\%)}: Depth of technical content and understanding.
\end{itemize}

\vspace{1cm}
\noindent\textbf{We hope this feedback helps identify areas for improvement. For further clarification, contact the marker.}

\vfill
\noindent \textbf{Marker:} \hfill \textbf{Date:}

\end{document}
